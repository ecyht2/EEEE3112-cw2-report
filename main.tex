\documentclass[12pt]{article}
\usepackage{nott-titlepage}
\usepackage{graphicx}
\usepackage{float}
\usepackage{amsmath}

\title{EEEE3112 Coursework 2 Inverter Design}
\author{Tan Hong Kai}
\date{May 13, 2024}
\studentid{20386501}
\module{EEEE3112 Power Electronic Applications and Control}
\department{Department of Electrical and Electronics}

\begin{document}
\maketitle

\section{Introduction}

This coursework explores the design of DC to AC inverter.
The inverter is a single-phase grid connected inverter facing a solar power network.
The design specification of the inverter is shown in table \ref{tab:design-spec}.

\begin{table}[H]
    \caption{Design Specification of the Inverter}
    \label{tab:design-spec}
    \centering{}
    \begin{tabular}{l l  l}
        \hline
        Specification/Term                  & Symbol      & Value  \\
        \hline
        DC Voltage                          & $V_{DC}$    & 600 V  \\
        AC RMS Voltage                      & $V_{AC}$    & 240 V  \\
        AC Current Ripple (peak-peak)       & $\Delta{I}$ & 0.5 A  \\
        Rated Power                         & $P$         & 3 kW   \\
        Inverter Switching Frequency        & $f_{sw}$    & 20 kHz \\
        Loss in the Inductor at Rated Power & $P_{LOSS}$  & 1\%P W \\
        Reactive Power                      & $Q_{AC}$    & 0 W    \\
        \hline
    \end{tabular}
\end{table}

\section{Methodology and Assumptions}

In this coursework, all the calculations are done in a Matlab live script.
Furthermore, the live script also contains the code for the control design and analysis.
Using a live script allows for quick calculations of values if any changes were made and the built-in markup makes the code organize.
Once the values for the inverter are calculated, the inverter is then modelled and simulated in PLECS to evaluate the performance.

\subsection{Inductor Design}

The first part of an inverter design is designing the inductor.
The inductor sits between the inverter and the AC grid.
It acts as a filter to the high frequency of the inverter.
Furthermore, the inductor acts as a load with low losses to allow for power transfer.

The main components that effects the inductor design is the AC current ripple $\Delta{I}$.
However, there are other components such as the switching period and DC input voltage which are usually fixed.
Formula \ref{eq:inductor} shows how the inductor size is calculated.
Where $T_{s}$ is the switching period $\frac{1}{f_{sw}}$.

\begin{equation} \label{eq:inductor}
    L = \frac{T_{s} V_{DC}}{8 \Delta{I}} = \frac{0.00005 * 600}{8 * 0.5} = 7.5 mH
\end{equation}

Inductors are not ideal and has some resistance in it.
In this coursework, the resistance of the inductor contributes to 1\% of the rated power.
It can be calculated using the power loss equation:

\begin{equation}
    \begin{aligned}
        P_{LOSS} & = I^{2} R                \\
        R        & = \frac{P_{LOSS}}{I^{2}}
    \end{aligned}
\end{equation}

Where:

\begin{equation}
    \begin{aligned}
        I      & = \frac{\sqrt{P^{2} + Q_{AC}^{2}}}{V_{AC}} = \frac{\sqrt{3000^{2} + 0^{2}}}{240} = 12.5 A \\
        \varphi & = -\arctan(\frac{Q_{AC}}{P_{AC}}) = -\arctan(0) = 0^{\circ{}}
    \end{aligned}
\end{equation}

Therefore:

\begin{equation}
    R = \frac{0.01 * 3000}{12.5^{2}} = 0.192 \Omega
\end{equation}

\section{Results and Discussion}

\end{document}
